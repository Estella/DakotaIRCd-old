\chapter{Operator Commands}
\label{commands}

\section{Network management commands}
\notebox{Note}{
	All commands and names are case insensitive.
	Parameters consisting of one or more separate letters,
	such as in MODE, STATS and WHO, are case sensitive.
}

\subsection{CONNECT}

\command{CONNECT} \userliteral{target} [\userliteral{port}] [\userliteral{source}]

	Initiate a connection attempt to server \userliteral{target}. If
	\userliteral{port} is given, connect to that port on the target,
	otherwise use the one given in \nolinkurl{ircd.conf}. If
	\userliteral{source} is given, tell that server to initiate the
	connection attempt, otherwise it will be made from the server you
	are attached to.

	To use the default port with \userliteral{source}, specify 0 for
	\userliteral{port}.

\subsection{SQUIT}

\command{SQUIT} \userliteral{server} [\userliteral{reason}]

	Closes down the link to \userliteral{server} from this side of the
	network. If \userliteral{reason} is given, it will be sent out in
	the server notices on both sides of the link.


\subsection{REHASH}

\command{REHASH} [\literal{CONFIG | BANS | DNS | MOTD | OMOTD | TKLINES | TDLINES | TXLINES |
	TRESVS | REJECTCACHE | HELP}] [\userliteral{server}]

	With no parameter given, CONFIG, MOTD, and OMOTD will be rehashed. The
	\userliteral{server} argument is a wildcard match of server names.

{\sc Parameters}

\noindent
\begin{description}
\item[{CONFIG}]
	Rereads \nolinkurl{ircd.conf}

\item[{BANS}]
	Rereads the ban database.

\item[{DNS}]
	Reread \nolinkurl{/etc/resolv.conf}.

\item[{MOTD}]
	Reload the MOTD file

\item[{OMOTD}]
	Reload the operator MOTD file

\item[{TKLINES}]
	Clears temporary K:lines.

\item[{TDLINES}]
	Clears temporary D:lines.

\item[{TXLINES}]
	Clears temporary X:lines.

\item[{TRESVS}]
	Clears temporary reservations.

\item[{REJECTCACHE}]
	Clears the client rejection cache.

\item[{HELP}]
	Refreshes the help system cache.
\end{description}

\subsection{RESTART}

\command{RESTART}
	\userliteral{server}

	Cause an immediate total shutdown of the IRC server, and restart from
	scratch as if it had just been executed.

	This reexecutes the ircd using the compiled-{}in path, visible as SPATH
	in INFO.

\subsection{DIE}

\command{DIE} \userliteral{server}
	Immediately terminate \userliteral{server}, after sending notices to all
	connected clients and servers

\subsection{SET}

\command{SET} [\literal{ADMINSTRING | AUTOCONN | AUTOCONNALL | FLOODCOUNT |
	IDENTTIMEOUT | MAX | OPERSTRING | SPAMNUM | SPAMTIME | SPLITMODE | SPLITNUM |
	SPLITUSERS}] \userliteral{value}

	The SET command sets a runtime-{}configurable value.

	Most of the \nolinkurl{ircd.conf} equivalents have a default\_ prefix
	and are only read on startup. SET is the only way to change these at
	run time.

	Most of the values can be queried by omitting \userliteral{value}.

\noindent
\begin{description}
\item[{ADMINSTRING}]
	Sets string shown in WHOIS for admins. (umodes +o and +a set, umode +S
	not set).

\item[{AUTOCONN}]
	Sets auto-{}connect on or off for a particular server. Takes two
	parameters, server name and new state.

	To see these values, use /stats c. Changes to this are lost on a
	rehash.

\item[{AUTOCONNALL}]
	Globally sets auto-{}connect on or off. If disabled, no automatic
	connections are done; if enabled, automatic connections are done
	following the rules for them.

\item[{FLOODCOUNT}]
	The number of lines allowed to be sent to a connection before
	throttling it due to flooding. Note that this variable is used for both
	channels and clients.

	For channels, op or voice overrides this; for users, IRC operator
	status or op or voice on a common channel overrides this.

\item[{IDENTTIMEOUT}]
	Timeout for requesting ident from a client.

\item[{MAX}]
	Sets the maximum number of connections to \userliteral{value}.

	This number cannot exceed maxconnections -{} MAX\_BUFFER.
	maxconnections is the rlimit for number of open files. MAX\_BUFFER is
	defined in config.h, normally 60.

	MAXCLIENTS is an alias for this.

\item[{OPERSTRING}]
	Sets string shown in WHOIS for opers (umode +o set, umodes +a and +S
	not set).

\item[{SPAMNUM}]
	Sets how many join/parts to channels constitutes a possible spambot.

\item[{SPAMTIME}]
	Below this time on a channel
	counts as a join/part as above.

\item[{SPLITMODE}]
	Sets splitmode to \userliteral{value}:
\noindent
\begin{description}
\item[{ON}]
	splitmode is permanently on

\item[{OFF}]
	splitmode is permanently off (default if no\_create\_on\_split and
	no\_join\_on\_split are disabled)

\item[{AUTO}]
	ircd chooses splitmode based on SPLITUSERS and SPLITNUM (default if
	no\_create\_on\_split or no\_join\_on\_split are enabled)
\end{description}

\item[{SPLITUSERS}]
	Sets the minimum amount of users needed to deactivate automatic
	splitmode.

\item[{SPLITNUM}]
	Sets the minimum amount of servers needed to deactivate automatic
	splitmode. Only servers that have finished bursting count for this.
\end{description}

\section{User management commands}
\label{usercommands}

\subsection{KILL}

\command{KILL}
	\userliteral{nick} [\userliteral{reason}]

	Disconnects the user with the given nick from the server they are
	connected to, with the reason given, if present, and broadcast a server
	notice announcing this.

	Your nick and the reason will appear on channels.

\subsection{CLOSE}

\command{CLOSE}
	Closes all connections from and to clients and servers who have not
	completed registering.

\subsection{KLINE}

\command{KLINE}
	[\userliteral{length}]
	[\userliteral{user}@\userliteral{host} |
	 \userliteral{user}@\userliteral{a.b.c.d}]
	[\literal{ON} \userliteral{servername}] [\userliteral{reason}]

	Adds a K:line to the ban database to ban the given user@host from
	using that server.

	If the optional parameter \userliteral{length} is given, the K:line will
	be temporary (i.e. it will not be stored on disk) and last that long in
	minutes.

	If an IP address is given, the ban will be against all hosts matching
	that IP regardless of DNS. The IP address can be given as a full
	address (192.168.0.1), as a CIDR mask (192.168.0.0\slash24), or as a
	glob (192.168.0.*).

	All clients matching the K:line will be disconnected from the server
	immediately.

	If a reason is specified, it will be sent to the client when they are
	disconnected, and whenever a connection is attempted which is banned.

	If the ON part is specified, the K:line is set on servers matching the
	given mask (provided a matching shared\{\} block exists there).
	Otherwise, if specified in a cluster\{\} block, the K:Line will be
	propagated across the network accordingly.

\subsection{UNKLINE}

\command{UNKLINKE}
	\userliteral{user}\literal{@}\userliteral{host}
	[\literal{ON} \userliteral{servername}]

	Will attempt to remove a K:line matching user@host from
	the ban database, and will flush a temporary K:line.

\subsection{XLINE}
\label{XLINES}

\command{XLINE} [\userliteral{length}] \userliteral{mask} [\literal{ON} \userliteral{servername}] [\literal{:}userliteral{reason}]

	Works similarly to KLINE, but matches against the real name field. The
	wildcards are * (any sequence), ? (any character), \# (a digit) and @
	(a letter); wildcard characters can be escaped with a backslash. The
	sequence \textbackslash{}s matches a space.

	All clients matching the X:line will be disconnected from the server
	immediately.

	The reason is never sent to users. Instead, they will be exited with
	`Bad user info'.

	If the ON part is specified, the X:line is set on servers matching the
	given mask (provided a matching shared\{\} block exists there).
	Otherwise, if specified in a cluster\{\} block, the X:line will be
	propagated across the network accordingly.

\subsection{UNXLINE}

\command{UNXLINE}
	\userliteral{mask} [\literal{ON} \userliteral{servername}]

	Will attempt to remove an X:line from the ban database, and will
	flush a temporary X:line.

\subsection{RESV}

\command{RESV}
	[\userliteral{length}] [\userliteral{channel} | \userliteral{mask}]
	[\literal{ON} \userliteral{servername}] [\literal{:}\userliteral{reason}]

	If used on a channel, `jupes' the channel locally. Joins to the
	channel will be disallowed and generate a server notice on +y, and
	users will not be able to send to the channel. Channel jupes cannot
	contain wildcards.

	If used on a nickname mask, prevents local users from using a nick
	matching the mask (the same wildcard characters as xlines). There is
	no way to exempt the initial nick from this.

	In neither case will current users of the nick or channel be kicked or
	disconnected.

	This facility is not designed to make certain nicks or channels
	oper-{}only.

	The reason is never sent to users.

	If the ON part is specified, the resv is set on servers matching the
	given mask (provided a matching shared\{\} block exists there).
	Otherwise, if specified in a cluster\{\} block, the resv will be
	propagated across the network accordingly.

\subsection{UNRESV}

\command{UNRESV} [\userliteral{channel} | \userliteral{mask}] [\literal{ON} \userliteral{servername}]

	Will attempt to remove a resv from the ban database, and will
	flush a temporary resv.

\subsection{DLINE}

\command{DLINE} [\userliteral{length}] \userliteral{a.b.c.d} [\literal{ON} \userliteral{servername}] [\literal{:}\userliteral{reason}]

	Adds a D:line to the ban database, which will deny any
	connections from the given IP address. The IP address can be given as a
	full address (192.168.0.1) or as a CIDR mask (192.168.0.0\slash24).

	If the optional parameter \userliteral{length} is given, the D:line
	will be temporary (i.e. it will not be stored on disk) and last that
	long in minutes.

	All clients matching the D:line will be disconnected from the server
	immediately.

	If a reason is specified, it will be sent to the client when they are
	disconnected, and, if dline\_reason is enabled, whenever a connection
	is attempted which is banned.

	D:lines are less load on a server, and may be more appropriate if
	somebody is flooding connections.

	If the ON part is specified, the D:line is set on servers matching the
	given mask (provided a matching shared\{\} block exists there, which is
	not the case by default). Otherwise, the D:Line will be set on the
	local server only.

	Only exempt\{\} blocks are exempt from D:lines. Being a server or having
	kline\_exempt in auth\{\} does \emph{not} exempt (different from
	K/G/X:lines).

\subsection{UNDLINE}

\command{UNDLINE} \userliteral{a.b.c.d} [\literal{ON} \userliteral{servername}]
	Will attempt to remove a D:line from the ban database, and will
	flush a temporary D:line.

\subsection{GRANT}

\command{GRANT} \userliteral{nick} \userliteral{privset}
	Grants the given user the set of privileges specified by privset. Note that
	this requires the oper:grant privilege.

	Any privileges granted by GRANT will be lost when the user disconnects. Users
	can also still regain oper if their oper block is in effect. It is highly
	advised, therefore, that this only be used as a temporary or transient measure.


\subsection{UNGRANT}

\command{UNGRANT} \userliteral{nick}
	Removes all oper privileges from a given user and removes their oper status.

\subsection{TESTGECOS}

\command{TESTGECOS} \userliteral{gecos}
	Looks up X:Lines matching the given gecos.

\subsection{TESTLINE}

\command{TESTLINE}
	[\userliteral{nick}!]
	[\userliteral{user}\literal{@}\userliteral{host} | \userliteral{a.b.c.d}]

	Looks up the given hostmask or IP address and reports back on any
	auth\{\} blocks, D: or K: lines found. If \userliteral{nick} is
	given, also searches for nick resvs.

	For temporary items the number of minutes until the item expires
	is shown (as opposed to the hit count in STATS q/Q/x/X).

	This command will not perform DNS lookups; for best
	results you must testline a host and its IP form.

\subsection{TESTMASK}

\command{TESTMASK}
	\userliteral{hostmask} [\userliteral{gecos}]

	Searches the network for users that match the hostmask and gecos given,
	returning the number of matching users on this server and other
	servers.

	The \userliteral{hostmask} is of the form user@host or user@ip/cidr
	with * and ? wildcards, optionally preceded by nick!.

	The \userliteral{gecos} field accepts the same wildcards as
	\hyperref[XLINES]{xlines}.

	The IP address checked against is 255.255.255.255 if the IP address is
	unknown (remote client on a TS5 server) or 0.0.0.0 if the IP address is
	hidden (auth\{\} spoof).

\subsection{LUSERS}

\command{LUSERS} [\userliteral{mask}] [\userliteral{nick} | \userliteral{server}]

	Shows various user and channel counts.

	The \userliteral{mask} parameter is obsolete but must be used when
	querying a remote server.

\subsection{TRACE}

\command{TRACE} [\userliteral{server} | \userliteral{nick}] [\userliteral{location}]
	With no argument or one argument which is the current server,
	TRACE gives a list of all connections to the current server
	and a summary of connection classes.

	With one argument which is another server, TRACE displays the path
	to the specified server, and all servers, opers and -{}i users
	on that server, along with a summary of connection classes.

	With one argument which is a client, TRACE displays the
	path to that client, and that client's information.

	If location is given, the command is executed on that server;
	no path is displayed.

	When listing connections, type, name and class is shown
	in addition to information depending on the type:

{\sc TRACE types}
\nopagebreak

\noindent
\begin{description}
\item[{Try.}]
	A server we are trying to make a TCP connection to.

\item[{H.S.}]
	A server we have established a TCP connection to, but is not
	yet registered.

\item[{????}]
	An incoming connection that has not yet registered as a user or a
	server (`unknown'). Shows the username, hostname, IP address and the
	time the connection has been open. It is possible that the ident or DNS
	lookups have not completed yet, and in any case no tildes are shown
	here. Unknown connections may not have a name yet.

\item[{User}]
	A registered unopered user. Shows the username, hostname, IP address,
	the time the client has not sent anything (as in STATS l) and the time
	the user has been idle (from PRIVMSG only, as in WHOIS).

\item[{Oper}]
	Like User, but opered.

\item[{Serv}]
	A registered server. Shows the number of servers and users reached via
	this link,who made this connection and the time the server has not sent
	anything.

\end{description}

\subsection{ETRACE}

\command{ETRACE}
	[\userliteral{nick}]

	Shows client information about the given target, or about all local
	clients if no target is specified.

\subsection{PRIVS}

\command{PRIVS} [\userliteral{nick}]
	Displays effective operator privileges for the specified nick, or for
	yourself if no nick is given. This includes all privileges from the
	operator block, the name of the operator block and those privileges
	from the auth block that have an effect after the initial connection.

	The exact output depends on the server the nick is on, see the matching
	version of this document. If the remote server does not support this
	extension, you will not receive a reply.

\subsection{MASKTRACE}

\command{MASKTRACE} \userliteral{hostmask} [\userliteral{gecos}]
	Searches the local server or network for users that match the hostmask
	and gecos given. Network searches require the oper\_spy privilege and
	an '!' before the hostmask. The matching works the same way as
	TESTMASK.

	The \userliteral{hostmask} is of the form user@host or user@ip/cidr with
	* and ? wildcards, optionally preceded by nick!.

	The \userliteral{gecos} field accepts the same wildcards as
	\hyperref[XLINES]{xlines}.

	The IP address field contains 255.255.255.255 if the IP address is
	unknown (remote client on a TS5 server) or 0 if the IP address is
	hidden (auth\{\} spoof).

\subsection{CHANTRACE}

\command{CHANTRACE} \userliteral{channel}
	Displays information about users in a channel. Opers with the
	oper\_spy privilege can get the information without being on the
	channel, by prefixing the channel name with an '!'.

	The IP address field contains 255.255.255.255 if the IP address is
	unknown (remote client on a TS5 server) or 0 if the IP address is
	hidden (auth\{\} spoof).

\subsection{SCAN}

\command{SCAN UMODES}
	\literal{+}\userliteral{modes}\literal{-}\userliteral{modes}
	[\literal{no-{}list} | \literal{list}] [\literal{global}]
	[\literal{list-{}max} \userliteral{number}]
	[\literal{mask} \userliteral{nick!user@host}]

	Searches the local server or network for users that have the umodes
	given with + and do not have the umodes given with -{}.	no-{}list
	disables the listing of matching users and only shows the count.

	list enables the listing (default).

	global extends the search to the entire network instead of local users
	only. list-{}max limits the listing of matching users to the given
	amount.	mask causes only users matching the given nick!user@host mask
	to be selected. Only the displayed host is considered, not the
	IP address or real host behind dynamic spoofs.

	The IP address field contains 255.255.255.255 if the IP address
	is unknown (remote client on a TS5 server) or 0 if the IP address
	is hidden (auth\{\} spoof).

	Network searches where a listing is given are operspy commands.

\subsection{CHGHOST}

\command{CHGHOST}
	\userliteral{nick} \userliteral{value}

	Set the hostname associated with a particular nick for the duration of
	this session.

\subsection{FORCEJOIN}

\command{FORCEJOIN} \userliteral{nick} \userliteral{channel}
	       Force a user to join a channel.

\notebox{Note}{
	       This requires extensions/m\_forcejoin.so.
}

\subsection{FORCENICK}

\command{FORCENICK} \userliteral{current-{}nick} \userliteral{new-{}nick}

	       Forcibly change a user's nick.

\notebox{Note}{
	       This requires extensions/m\_forcenick.so.
}

\subsection{FORCEPART}

\command{FORCEPART} \userliteral{nick} \userliteral{channel}
	       Force a user to part a channel.

\notebox{Note}{
	       This requires extensions/m\_forcepart.so.
}


\subsection{FORCEQUIT}

\command{FORCEQUIT} \userliteral{nick} \userliteral{reason}
	       Force a user to quit.

\notebox{Note}{
	       This requires extensions/m\_forcequit.so (which lives in unsupported).
}

\section{Miscellaneous commands}
\label{misccommands}

\subsection{ADMIN}

\command{ADMIN} [\userliteral{nick} | \userliteral{server}]

	Shows the information in the admin\{\} block.


\subsection{INFO}

\command{INFO} [\userliteral{nick} | \userliteral{server}]

	Shows information about the authors of the IRC server, and
	some information about this server instance.
	Opers also get a list of configuration options.


\subsection{TIME}

\command{TIME} [\userliteral{nick} | \userliteral{server}]

	Shows the local time on the given server, in a human-{}readable format.


\subsection{VERSION}

\command{VERSION}
	[\userliteral{nick} | \userliteral{server}]

	Shows version information, a few compile/config options,
	the SID and the 005 numerics.
	The 005 numeric will be remapped to 105 for remote requests.


\subsection{STATS}

\command{STATS} [\userliteral{type}] [\userliteral{nick} | \userliteral{server}]

	Display various statistics and configuration information.


{\sc Values for \userliteral{type}}
\nopagebreak

\noindent
\begin{description}
\item[{A}]
	Show DNS servers

\item[{b}]
	Show active nick delays

\item[{B}]
	Show hash statistics

\item[{c}]
	Show connect blocks

\item[{d}]
	Show temporary D:lines

\item[{D}]
	Show permanent D:lines

\item[{e}]
	Show exempt blocks (exceptions to D:lines)

\item[{E}]
	Show events

\item[{f}]
	Show file descriptors

\item[{h}]
	Show hub\_mask/leaf\_mask

\item[{i}]
	Show auth blocks, or matched auth blocks

\item[{k}]
	Show temporary K:lines, or matched K:lines

\item[{K}]
	Show permanent K:lines, or matched K:lines

\item[{l}]
	Show hostname and link information about the given nick. With a server
	name, show information about opers and servers on that server; opers
	get information about all local connections if they query their own
	server. No hostname is shown for server connections.

\item[{L}]
	Like l, but show IP address instead of hostname

\item[{m}]
	Show commands and their usage statistics (total counts, total bytes,
	counts from server connections)

\item[{n}]
	Show blacklist blocks (DNS blacklists) with hit counts since last
	rehash and (parenthesized) reference counts. The reference count shows
	how many clients are waiting on a lookup of this blacklist or have been
	found and are waiting on registration to complete.

\item[{o}]
	Show operator blocks

\item[{O}]
	Show privset blocks

\item[{p}]
	Show logged on network operators which are not set AWAY.

\item[{P}]
	Show listen blocks (ports)

\item[{q}]
	Show temporarily resv'ed nicks and channels with hit counts

\item[{Q}]
	Show permanently resv'ed nicks and channels with hit counts since last
	rehash bans

\item[{r}]
	Show resource usage by the ircd

\item[{t}]
	Show generic server statistics about local connections

\item[{u}]
	Show server uptime

\item[{U}]
	Show shared (c), cluster (C) and service (s) blocks

\item[{v}]
	Show connected servers and brief status

\item[{x}]
	Show temporary X:lines with hit counts

\item[{X}]
	Show permanent X:lines with hit counts since last rehash bans

\item[{y}]
	Show class blocks

\item[{z}]
	Show memory usage statistics

\item[{Z}]
	Show ziplinks statistics

\item[{?}]
	Show connected servers and link information about them
\end{description}

\subsection{WALLOPS}

\command{WALLOPS} \literal{:}\userliteral{message}

	Sends a WALLOPS message to all users who have the +w umode set. This is
	for things you don't mind the whole network knowing about.


\subsection{OPERWALL}

\command{OPERWALL} \literal{:}\userliteral{message}
	Sends an OPERWALL message to all opers who have the +z umode set. +z is
	restricted, OPERWALL should be considered private communications.
