\chapter{Channel modes}
\label{cmodes}

\section{Meanings of channel modes}

\subsection{+a, channel admin}
	Channel admins are immune to being kicked by anyone other than other
	admins and channel founders.  They are able to grant admin status to
	other users as well as revoke it from other admins.  This status
	also inherits all channel operator powers, like changing modes, setting
	topics, etc.

	In SporksIRCD, this is represented with an exclamation mark (!).

\subsection{+A, admins only}
	If this mode is set, only oper with the oper:admin flag may join.

	\notebox{Note}{
		This requires \nolinkurl{extensions/chm\_adminonly.so}.
	}

\subsection{+b, channel ban}
	Bans take one parameter which can take several forms. The most common
	form is +b nick!user@host. The wildcards * and ? are allowed, matching
	zero-{}or-{}more, and exactly-{}one characters respectively. The masks
	will be trimmed to fit the maximum allowable length for the relevant
	element. Bans are also checked against the IP address, even if it
	resolved or is spoofed.

	CIDR is supported, like *!*@10.0.0.0/8. This is most useful with
	IPv6.

	Bans are not checked against the real hostname behind any kind of
	spoof, except if host mangling is in use (e.g.
	\nolinkurl{extensions/usm\_ip\_cloaking.so}): if the user's host is mangled,
	their real hostname is checked additionally, and if a user has no
	spoof but could enable mangling, the mangled form of their hostname is
	checked additionally. Hence, it is not possible to evade bans by
	toggling host mangling.

	The second form (extban) is +b \$type or +b \$type:data. type is a
	single character (case insensitive) indicating the type of match,
	optionally preceded by a tilde (\textasciitilde{}) to negate the
	comparison. data depends on type. Each type is loaded as a module. The
	available types (if any) are listed in the EXTBAN token of the 005
	(RPL\_ISUPPORT) numeric. See \nolinkurl{doc/extban.txt} in the source
	distribution for more information.

	If no parameter is given, the list of bans is returned. All users can
	use this form. The plus sign should also be omitted.

	Matching users will not be allowed to join the channel or knock on it.
	If they are already on the channel, they may not send to it or change
	their nick.


\subsection{+c, colour filter}
	This cmode activates the colour filter for the channel. This filters
	out bold, underline, reverse video, beeps, mIRC colour codes, and ANSI
	escapes. Note that escape sequences will usually leave cruft sent to
	the channel, just without the escape characters themselves.

	\notebox{Note}{
		This requires \nolinkurl{extensions/chm\_nocolour.so}.
	}

\subsection{+C, no CTCP}
	This mode blocks all CTCP messages except for CTCP ACTION from being
	sent to a channel.

	\notebox{Note}{
		This requires \nolinkurl{extensions/chm\_noctcp.so}.
	}

\subsection{+d, no nick changes}
	This mode blocks all users on a channel from changing their nicks.

\subsection{+e, ban exemption}
	This mode takes one parameter of the same form as bans, which overrides
	+b and +q bans for all clients it matches.

	This can be useful if it is necessary to ban an entire ISP due to
	persistent abuse, but some users from that ISP should still be allowed
	in. For example: /mode \#channel +be *!*@*.example.com
	*!*someuser@host3.example.com

	Only channel operators can see +e changes or request the list.

\subsection{+f, channel forwarding}
	This mode takes one parameter, the name of a channel (+f \#channel). If
	the channel also has the +i cmode set, and somebody attempts to join
	without either being expliticly invited, or having an invex (+I), then
	they will instead join the channel named in the mode parameter. The
	client will also be sent a 470 numeric giving the original and target
	channels.

	Users are similarly forwarded if the +j cmode is set and their attempt
	to join is throttled, if +l is set and there are already too many users
	in the channel or if +r is set and they are not identified.

	Forwards may only be set to +F channels, or to channels the setter has
	ops in.

	Without parameter (/mode \#channel f or /mode \#channel +f) the forward
	channel is returned. This form also works off channel.

\subsection{+F, allow anybody to forward to this}
	When this mode is set, anybody may set a forward from a channel they
	have ops in to this channel. Otherwise they have to have ops in this
	channel.

\subsection{+g, allow anybody to invite}
	When this mode is set, anybody may use the INVITE command on the
	channel in question. When it is unset, only channel operators may use
	the INVITE command.

	When this mode is set together with +i, +j, +l or +r, all channel
	members can influence who can join.

\subsection{+G, no all-{}caps messages}
	This mode blocks all messages that have more than a set amount of
	capital letters. The percentage is controlled by the caps\_threshold
	option.

	This requires \nolinkurl{extensions/chm\_blockcaps.so}.

\subsection{+h, half-operator}
	Half-operators have most of the powers of full channel operators, like
	setting modes, kicking users, banning, et cetera, but may not grant others
	+h, revoke it from others, or kick anyone who is not a regular user, a
	voiced user, or another halfop.

	Like in most other ircds with +h, this is represented with a percent sign (\%).

\subsection{+i, invite only}
	When this cmode is set, no client can join the channel unless they have
	an invex (+I) or are invited with the INVITE command.

\subsection{+I, invite exception (invex)}
	This mode takes one parameter of the same form as bans. Matching
	clients do not need to be invited to join the channel when it is
	invite-{}only (+i). Unlike the INVITE command, this does not override
	+j, +l and +r.

	Only channel operators can see +I changes or request the list.

\subsection{+j, join throttling}
	This mode takes one parameter of the form
	\userliteral{n}:\userliteral{t}, where \userliteral{n} and
	\userliteral{t} are positive integers. Only \userliteral{n} users may
	join in each period of \userliteral{t} seconds.

	Invited users can join regardless of +j, but are counted as normal.

	Due to propagation delays between servers, more users may be
	able to join (by racing for the last slot on each server).

\subsection{+J, no autorejoin on kick}
	This mode prevents users from automatically rejoining a channel after a
	KICK.

\subsection{+k, key (channel password)}
	Taking one parameter, when set, this mode requires a user to supply the
	key in order to join the channel: /JOIN \#channel key.

\subsection{+K, no repeat messages}
	When set, any  message will be blocked if it is the same as the
	previous one sent.

	\notebox{Note}{
		This requires \nolinkurl{extensions/chm\_norepeat.so}.
	}

\subsection{+l, channel member limit}
	Takes one numeric parameter, the number of users which are allowed to
	be in the channel before further joins are blocked. Invited users may
	join regardless.

	Due to propagation delays between servers, more users may be able to
	join (by racing for the last slot on each server).

\subsection{+L, large ban list}
	Channels with this mode will be allowed larger banlists (by default,
	500 instead of 50 entries for +b, +q, +e and +I together). Only network
	operators with resv privilege may set this mode.

\subsection{+m, moderated}
	When a channel is set +m, only users with +o or +v on the channel can
	send to it.

	Users can still knock on the channel or change their nick.

\subsection{+M, immune}
	When this mode is set, opers cannot be kicked from the channel.
	Any admin may set this mode on a channel even without having operator
	status in it.

\subsection{+n, no external messages}
	When set, this mode prevents users from sending to the channel without
	being in it themselves. This is recommended.

\subsection{+N, allow roleplay}
	When set, this mode allows users to use roleplay commands, such as NPC,
	NPCA, SCENE, and FACTION.

	This mode is not available unless \nolinkurl{extensions/m\_roleplay.so} is loaded.

\subsection{+o, channel operator}
	This mode takes one parameter, a nick, and grants or removes channel
	operator privilege to that user. Channel operators have full control
	over the channel, having the ability to set all channel modes except
	+L and +P, and kick users. Like voiced users, channel operators can
	always send to the channel, overriding +b, +m and +q modes and the
	per-{}channel flood limit.
	In most clients channel operators are marked with an '@' sign.

	The privilege is lost if the user leaves the channel or server
	in any way.

	Most networks will run channel registration services (e.g. ChanServ)
	which ensure the founder (and users designated by the founder) can
	always gain channel operator privileges and provide some features to
	manage the channel.

\subsection{+O, opers only}
	This mode blocks all users who are not opers from joining.

	\notebox{Note}{
		This requires \nolinkurl{extensions/chm\_operonly.so}.
	}

\subsection{+p, paranoid channel}
	When set, the KNOCK command cannot be used on the channel to request an
	invite, and users will not be shown the	channel in WHOIS replies unless
	they are on it. Unlike in traditional IRC, +p and +s can be set
	together.

\subsection{+P, permanent channel}
	Channels with this mode (which is accessible only to network operators
	with resv privilege) set will not be destroyed when the last user
	leaves.

	This makes it less likely modes, bans and the topic will be lost and
	makes it harder to abuse network splits, but also causes more unwanted
	restoring of old modes, bans and topics after long splits.

\subsection{+q, quiet}

	This mode behaves exactly like +b (ban), except that the user may still
	join the channel. The net effect is that they cannot knock on the
	channel, send to the channel or change their nick while on channel.

\subsection{+Q, block forwarded users}

	Channels with this mode set are not valid targets for forwarding. Any
	attempt to forward to this channel will be ignored, and the user will
	be handled as if the attempt was never made (by	sending them the
	relevant error message).

	This does not affect the ability to set +f.

\subsection{+r, block unidentified}
	When set, this mode prevents unidentified users from joining. Invited
	users can still join.

\subsection{+s, secret channel}
	When set, this mode prevents the channel from appearing in the output
	of the LIST, WHO and WHOIS command by users who are not on it. Also
	the server will refuse to answer WHO, NAMES, TOPIC and LIST queries
	from users not on the channel.

\subsection{+S, ssl only}
	When set, only users connecting with SSL may join.

	This requires \nolinkurl{extensions/chm\_ssonly.so}.

\subsection{+t, topic limit}
	When set, this mode prevents users who are not channel operators from
	changing the topic.

\subsection{+T, no notices}
	This mode blocks all NOTICES from being sent to a channel.

	This requires \nolinkurl{extensions/chm\_nonotices.so}.

\subsection{+u, channel founder}
        Channel founders are immune to being kicked by anyone except other
	channel founders.  They are able to grant founder status to other users
	as well as revoke it from other founders.  This status also inherits all
	channel admin and channel operator powers.

	In SporksIRCD, this is represented with a tilde (~).

\subsection{+v, voice}
	This mode takes one parameter, a nick, and grants or removes voice
	privilege to that user. Voiced users can always send to the channel,
	overriding +b, +m and +q modes and the per-{}channel flood limit. In
	most clients voiced users are marked with a plus sign.


	The privilege is lost if the user leaves the channel or server in any
	way.

\subsection{+W, enable censors to be set}
	When +W is set, +y modes may be set by channel operators and above.
	Note that +W can only be set by an IRC operator. This is because the
	censor filter could bog down the ircd if used on too many channels
	(inspircd suffers from this problem, but more so).

	If \nolinkurl{extensions/chm\_censor\_operonly.so} is not loaded, then
	censors may be set on channels without needing +W.

\subsection{+y, censor}
	This mode allows custom censors to be set on a channel (similar to
	inspircd's +g). There are some key differences:
	\begin{itemize}

	\item{% <to>
		If extensions/chm\_censor\_operonly.so is loaded, censors may
		only be set if +W is set (an oper-{}only mode).
	}

	\item{% <to>
		Censors do no CTCP filtering. This is a feature. It can be used
		to block specific CTCP types such as DCC.
	}

	\item{% <to>
		Channel operators may be made exempt from censors via the
		exemptchanops configuration option
	}

	\end{itemize}

	This requires \nolinkurl{extensions/chm\_censor.so}.

\subsection{+z, reduced moderation}
	When +z is set, the effects of +m, +b and +q are relaxed. For each
	message, if that message would normally be blocked by one of these
	modes, it is instead sent to all channel operators. This is intended
	for use in moderated debates.

	Note that +n is unaffected by this. To silence a given user completely,
	remove them from the channel.

